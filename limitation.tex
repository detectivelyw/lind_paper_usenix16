\section{Limitations and Future Work}
\label{sec.limitation}

Though Lind shows potential as an effective method for protecting privileged code,
further investigation is required to address issues that were not included in our
initial study.
%We discuss the limitations of Lind in this section.
First, our metric relies on identifying lines of code in the kernel that are
executed when running applications in secure systems.
Our current study only focuses on OS kernel protection, while the area of hypervisor-based
VMs is also very interesting and worth looking into.
It is not clear if
existing hypervisors are small and compact enough to consider all the
identified paths traversing off executing applications as common paths or
whether there is still substantial benefit from our metric.
In %our
future work we will %explore
investigate if the bug metric works well for %operating system virtual machines with a hypervisor.
hypervisor-based VMs.

A second limitation stems from our criteria for determining if a kernel trace is
safe or risky. In our study, the criteria involved checking the trace against a list
 f historical kernel bugs.
 %However, not all bugs can be accurately checked in this manner.
But due to the nature of bugs our proposed metric can not assure if a portion of code
is absolutely a safe or a risky area in kernel.
So we are just trying to reduce the chances of triggering underlying kernel bugs, but can 
not promise to avoid all bugs.
For example, bugs that are caused
by a race condition cannot be identified by directly checking if certain lines of
code have been executed. For complicated bugs that involve defects in the internal
kernel data structures, or require complex triggering conditions across multiple
kernel paths, our metric will not be able to accurately determine whether or not
those bugs have been triggered. In such cases, more complex metrics might be needed
 for detection.

%While Lind executes programs like Apache and Tor, it does not
%support every system call or every possible set of arguments. For example,
%symbolic links are not supported.  While the SafePOSIX implementation could
%be extended to do so, we leave this for future work. Also, as mentioned in
%the previous section,


%There were also some avenues we intentionally excluded in this initial
%study that could form the basis for interesting research projects in the
%future. First, we chose not to explore bugs within the applications
%themselves.
%\cappos{move to motivation / threat model.}

Lind has not been optimized for performance, so the results we present here should be taken as a baseline.
%of what is possible.
We would like to explore what existing OS VM optimizations can be safely applied
and what their impact might be.

We would also like to test our metric in other popular operating systems, such as Windows and Mac OS.
Our experiments were limited to Linux kernel 3.14.1 and some of the typical virtualization systems that existed in Linux.
It would be interesting
%and beneficial to the advance of secure systems
if similar tests could be run in other widely-used operating systems. In particular, it would be interesting to see if
having the host and guest VM run different operating systems would produce different security impacts.
