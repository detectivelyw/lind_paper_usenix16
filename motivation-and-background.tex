\section{Motivation and Background}
\label{sec.motivation-and-background}

This section presents some critical background information
about the function of the OS kernel and why ensuring its security is so important.
A model of the form a potential threat could take is also provided.
%caused by an inability to identify which portions of the kernel are risky.

\subsection{Function and Design of the OS Kernel}

\yiwen{This subsection title 2.1 seems not very accurate to me. 
We are not talking about the ``function'' or ``design'' of an OS kernel, 
but rather the potential risks that a kernel could be facing.}

The kernel is responsible for
managing input and output requests from servers and converting this information
into system calls to execute operations. User applications depend directly on
their ability to access critical resources, such as memory, I/O, or CPU, from the
kernel, which provides interfaces to serve these requests, Unfortunately, these
types of operations bring the kernel in contact, with complex and potentially
malicious programs. OS kernels are vulnerable to adversarial attacks, which have
increased in frequency over time.
In 2014, 215 vulnerabilities in all types of kernels were reported~\cite{NVD},
of which 125 were in the Linux kernel alone.\lois{These two statements are related,
 but they are not cause and effect. The \# of vulnerabilities does not directly correlate
 to the number of attacks. The vulnerabilities are in the code, right?}
With the constant addition of new features~\cite{Metrics-13}, the kernel's
attack surface keeps increasing For example, the Linux kernel grew from 6.6 MLOC
in v2.6.11 (March 2005) to 16.9 MLOC in v3.10 (June 2013)~\cite{Linux-13}. A
common assumption is that there is a connection between the size of the attack
surface and a kernel's potential vulnerability, though this has not been proved
conclusively.
\lois{I found a paper that studies this relationship. It
probably should be cited. I'll provide that information offline.I also think
this paragraph needs at least one more sentence emphasizing the difficulty of
protecting the kernel.}

%\cappos{Likely cut this.  I think it is excessive.}

%Such a huge kernel codebase has lead to excessive exploitation, where it has been plagued by a number of common software flaws.
%These flaws have raised serious security concerns and caused severe damage to systems.
%\lois{Are these sentences needed? Is this what Justin thought was excessive? If so, I agree.}
%The common vulnerability and exposure (CVE) database reports indicate stack and heap buffer overflow vulnerabilities
%have been used to launch denial of service attacks through crashing the systems  (CVE-2013-2892).
%~\cite{CVE-2013-2892}, Other vulnerabilities such as the execution of arbitrary code (CVE-2009-3234) %~\cite{CVE-2009-3234},
%or to allow local users to gain privileges via a crafted application (CVE-2013-1828) are also among the reports.
%~\cite{CVE-2013-1828}. Memory disclosure vulnerabilities (CVE-2009-3002) have been also exploited to
%allow local users to read the contents of some kernel memory locations
%~\cite{CVE-2009-3002} , or to obtain potentially sensitive information from kernel stack memory (CVE-2010-4073). %~\cite{CVE-2010-4073}.
%Attackers have also used use-after-free vulnerability to gain kernel privileges, i.e. CVE-2013-4343.
%~\cite{CVE-2013-4343}. \lois{If this is what Justin thought was excessive, I would still agree.
%It is too much of a "laundry list" of random examples, with no specifics.}

%The number of kernel vulnerabilities and their potential for exploitation,
%plus the fact that user applications rely on the kernel to execute
%programs,
%present a compelling motive for designing securer systems that can run
%applications with a better degree of safety. \yanyan{this paragraph does
%not say anything new.}

\subsection{Threat Model}

%The primary =secure system is to restrict a program to a subset of privileges.
%Most systems mediate this access to
%the underlying operating system through a set of functions.
One of the great risks of a kernel exploitation is the access
it grants a potential attacker to privileged code. In most systems, the kernel
mediates access to the underlying system. Triggering kernel vulnerabilities
could extend access privileges
to attackers,thus sacrificing any intended protection~\cite{Repy-10}.

%\gholami{I doubt this is the goal of the system. I think system goals are CIA
%properties that can be assured through confinement for example?}.

With what is known about the vulnerability of kernel code, we make the
following assumptions about potential attacks:

\begin{enumerate}
\item  Bugs may exist in any complex code, including any security systems
placed between the kernel and the user applications. These flaws can be triggered
both intentionally by a malicious party, or unintentionally through contact with
untrusted programs.

\item An attacker also has the ability to execute code inside
of a virtualization system, such as an operating system VM or library OS.

\item Once vulnerabilities are triggered within kernel code, an attacker can gain the
privileged access delegated to this part of kernel. This gives an attacker unrestricted
access to the system.

\end{enumerate}

To respond to such a threat, it is feasible to build a functionality that
possesses few vulnerabilities, as long as the codebase is kept to a minimum.

%\cappos{How do I explainthe sandbox TCB???  How is this different from a hypervisor?}