\section{Evaluation}
\label{sec.evaluation}

We designed and conducted a set of experiments based around four fundamental questions to evaluate the performance of Lind:

\begin{itemize}
\item How does the security of Lind compare to other virtualized environments?
(\S{\ref{Linux-Kernel-Bug-Test-and-Evaluation}})

\item How much of the underlying kernel is reachable in different
virtualization systems?
(\S{\ref{Reachable-Kernel-Trace-Analysis-for-Different-Virtualization-Systems}})

\item If the SafePOSIX implementation has bugs, how much more of the kernel is
reachable?
(\S{\ref{Reachable-Kernel-Trace-Analysis-for-Repy-Sandbox}})

\item What happens if the Repy sandbox kernel has bugs?
(\S{\ref{Sandbox-Kernel-Bugs}})

\item What is the overhead of Lind in practice?
(\S{\ref{Performance-Evaluation}})
\end{itemize}

This section describes the experiments that we designed to answer these
question. It also argues how the results support the merits of our secure design and its prototype Lind. 

\subsection{Evaluation Methodology}

Our evaluation strategy was to directly compare the performance of Lind
against 
four other existing virtualization systems\textendash VirtualBox, VMWare
Workstation, 
Docker, and Graphene. We also compared it against Native Linux and used 
those results as a baseline for evaluation. Because Native Linux is the
original OS, 
without virtualization and additional protection, this data helps to
clarify the security benefits of Lind, 
as well as whatever performance overhead costs the system may incur.

\textbf{Experimental setup.}
We conducted our experiments under Linux kernel 3.14.1, using the following
protocols:

\begin{itemize}
\item We identified and examined a list of  69 historical bugs that have
specifically 
targeted Linux kernel 3.14.1 \cite{CVE-Datasource}. Through analyzing
security kernel patches for those bugs, 
we identified the lines of code in the kernel that correspond to each of
one.

\item In order to test if a bug is triggerable, we created or located C
code to 
exploit each of the kernel bugs \cite{Exploit-Database}. We were able to
run and obtain results for 
35 out of the 69 bugs in our experiments. For the rest of the bugs, 
we could not find code that would trigger them effectively at that
moment. 
In addition, it was very difficult to accurately determine if more complex
bugs 
were triggered or not. We leave further study and analysis of those bugs to
future work.

\item We compiled and ran the exploit C code under each virtualization
system to 
obtain their kernel traces, and then used our kernel trace safety metric to
determine 
if a specific bug was triggered inside that virtualization system. 

\item Lastly, to  analyze the reachable kernel paths for each of the
virtualization systems, 
we conducted system call fuzzing (similar to what we did in \S{\ref{sec.metric}}) to obtain
the kernel trace in each system. 
We repeated the system call fuzzing within Repy as well to obtain its
kernel trace. 
\end{itemize}

\subsection{Comparison Results and Analysis}

Since our primary goal was to build a secure system, our core evaluation
was conducted primarily 
from a security perspective
(\S{\ref{Linux-Kernel-Bug-Test-and-Evaluation}} --
%\S{\ref{Reachable-Kernel-Trace-Analysis-for-Different-Virtualization-Systems}}, 
\S{\ref{Reachable-Kernel-Trace-Analysis-for-Repy-Sandbox}}). 
We also conducted an evaluation of the performance overhead of Lind
(\S{\ref{Performance-Evaluation}}) 
to obtain some understanding of its potential overall efficacy in
real-world settings.

\subsubsection{Linux Kernel Bug Test and Evaluation}
\label{Linux-Kernel-Bug-Test-and-Evaluation}

We tested 35 Linux kernel bugs in Native Linux, VirtualBox, VMWare
Workstation, Docker, Graphene, 
and Lind, to evaluate if any of them can be triggered. The kernel bugs
examined 
are capable of causing serious security problems. For example, 
the CVE-2014-8989 bug allows local users to bypass intended file
permissions by leveraging a POSIX ACL. 
When running applications, the potential risk of triggering some of these
kernel bugs here, 
and possibly many more bugs outside our list, is a severe problem that
users should be concerned about.

\begin{table*}%[!ht]
\scriptsize
\centering
\caption {Linux Kernel Bugs, and Vulnerabilities in Different
Virtualization Systems 
({\color{red}\ding{51}}: vulnerability triggered; \ding{55}: vulnerability
not triggered).\cappos{It may be useful to add NaCl.  I suppose one could
argue that NaCl is really what is providing protection or that NaCl is good
enough.}}
\begin{tabular}{|l|c|c|c|c|c|c|}\hline
\textbf{Vulnerability}    &  \textbf{Native Linux}  &  \textbf{VirtualBox}
&  \textbf{VMWare Workstation}
 & \textbf{Docker} & \textbf{Graphene} & \textbf{Lind} \\
\hline
 CVE-2015-5706 & {\color{red}\ding{51}} & {\color{red}\ding{51}} &
{\color{red}\ding{51}} & {\color{red}\ding{51}} & {\color{red}\ding{51}} &
\ding{55}  \\
 CVE-2015-0239 & {\color{red}\ding{51}} & {\color{red}\ding{51}} &
{\color{red}\ding{51}} & \ding{55} & \ding{55}  & \ding{55}  \\
 CVE-2014-9584 & {\color{red}\ding{51}} & \ding{55}  & \ding{55}  &
\ding{55} & \ding{55}  & \ding{55}  \\
 CVE-2014-9529 & {\color{red}\ding{51}} & {\color{red}\ding{51}}  &
\ding{55}  & \ding{55} & \ding{55}  & \ding{55}  \\
 CVE-2014-9322 & {\color{red}\ding{51}} & {\color{red}\ding{51}}  &
\ding{55}  & {\color{red}\ding{51}} & {\color{red}\ding{51}}  & \ding{55}
\\
 CVE-2014-9090 & {\color{red}\ding{51}} & \ding{55}  & \ding{55}  &
\ding{55} & \ding{55}  & \ding{55}  \\
 CVE-2014-8989 & {\color{red}\ding{51}} & {\color{red}\ding{51}} &
{\color{red}\ding{51}} & {\color{red}\ding{51}} & {\color{red}\ding{51}} &
\ding{55}  \\
 CVE-2014-8559 & {\color{red}\ding{51}} & \ding{55}  & \ding{55}  &
\ding{55} & \ding{55}  & \ding{55}  \\
 CVE-2014-8369 & {\color{red}\ding{51}} & \ding{55}  & \ding{55}  &
\ding{55} & \ding{55}  & \ding{55}  \\
 CVE-2014-8160 & {\color{red}\ding{51}} & {\color{red}\ding{51}} &
{\color{red}\ding{51}} & \ding{55} & \ding{55}  & \ding{55}  \\
 CVE-2014-8134 & {\color{red}\ding{51}} & {\color{red}\ding{51}} &
{\color{red}\ding{51}} & \ding{55} & {\color{red}\ding{51}}  & \ding{55}
\\
 CVE-2014-8133 & {\color{red}\ding{51}} & {\color{red}\ding{51}}  &
\ding{55}  & \ding{55} & \ding{55}  & \ding{55}  \\
 CVE-2014-8086 & {\color{red}\ding{51}} & {\color{red}\ding{51}} &
{\color{red}\ding{51}} & {\color{red}\ding{51}} & \ding{55} & \ding{55}  \\
 CVE-2014-7975 & {\color{red}\ding{51}} & \ding{55}  & \ding{55}  &
\ding{55} & \ding{55}  & \ding{55}  \\
 CVE-2014-7970 & {\color{red}\ding{51}} & \ding{55}  & \ding{55}  &
\ding{55} & \ding{55}  & \ding{55}  \\
 CVE-2014-7842 & {\color{red}\ding{51}} & \ding{55}  & \ding{55}  &
\ding{55} & \ding{55}  & \ding{55}  \\
 CVE-2014-7826 & {\color{red}\ding{51}} & {\color{red}\ding{51}} &
{\color{red}\ding{51}} & \ding{55} & {\color{red}\ding{51}}  & \ding{55}
\\
 CVE-2014-7825 & {\color{red}\ding{51}} & {\color{red}\ding{51}} &
{\color{red}\ding{51}} & \ding{55} & {\color{red}\ding{51}}  & \ding{55}
\\
 CVE-2014-7283 & {\color{red}\ding{51}} & \ding{55}  & \ding{55}  &
\ding{55} & \ding{55}  & \ding{55}  \\
 CVE-2014-5207 & {\color{red}\ding{51}} & \ding{55}  & \ding{55}  &
\ding{55} & \ding{55}  & \ding{55}  \\
 CVE-2014-5206 & {\color{red}\ding{51}} & \ding{55}  &
{\color{red}\ding{51}}  & {\color{red}\ding{51}}& \ding{55}  & \ding{55}
\\
 CVE-2014-5045 & {\color{red}\ding{51}} & \ding{55}  & \ding{55}  &
\ding{55} & \ding{55}  & \ding{55}  \\
 CVE-2014-4943 & {\color{red}\ding{51}} & \ding{55}  & \ding{55}  &
\ding{55} & \ding{55}  & \ding{55}  \\
 CVE-2014-4667 & {\color{red}\ding{51}} & \ding{55}  & \ding{55}  &
\ding{55} & {\color{red}\ding{51}}  & \ding{55}  \\
 CVE-2014-4508 & {\color{red}\ding{51}} & \ding{55}  & \ding{55}  &
\ding{55} & \ding{55}  & \ding{55}  \\
 CVE-2014-4171 & {\color{red}\ding{51}} & {\color{red}\ding{51}} &
{\color{red}\ding{51}} & {\color{red}\ding{51}} & {\color{red}\ding{51}} &
{\color{red}\ding{51}}  \\
 CVE-2014-4157 & {\color{red}\ding{51}} & \ding{55}  & \ding{55}  &
\ding{55} & \ding{55}  & \ding{55}  \\
 CVE-2014-4014 & {\color{red}\ding{51}} & \ding{55}  &
{\color{red}\ding{51}}  & {\color{red}\ding{51}} & \ding{55}  & \ding{55}
\\
 CVE-2014-3940 & {\color{red}\ding{51}} & {\color{red}\ding{51}}  &
\ding{55}  & {\color{red}\ding{51}}& \ding{55}  & \ding{55}  \\
 CVE-2014-3917 & {\color{red}\ding{51}} & {\color{red}\ding{51}}  &
\ding{55}  & \ding{55} & \ding{55}  & \ding{55}  \\
 CVE-2014-3153 & {\color{red}\ding{51}} & \ding{55}  & \ding{55}  &
\ding{55} & \ding{55}  & \ding{55}  \\
 CVE-2014-3144 & {\color{red}\ding{51}} & \ding{55}  & \ding{55}  &
\ding{55} & \ding{55}  & \ding{55}  \\
 CVE-2014-3122 & {\color{red}\ding{51}} & \ding{55}  & \ding{55}  &
\ding{55} & \ding{55}  & \ding{55}  \\
 CVE-2014-2851 & {\color{red}\ding{51}} & \ding{55}  & \ding{55}  &
\ding{55} & \ding{55}  & \ding{55}  \\
 CVE-2014-0206 & {\color{red}\ding{51}} & \ding{55}  & \ding{55}  &
\ding{55} & \ding{55}  & \ding{55}  \\
\hline
 {\bf Vulnerabilities Triggered} & {\bf 35/35 (100\% )} & {\bf 14/35 (40\%)} & 
 {\bf 11/35 (31.4\%)}  & {\bf 8/35 (22.9\%)} & {\bf 8/35 (22.9\%)}  & {\bf 1/35 (2.9\%)}  \\
\hline
\end{tabular}
\label{table:trigger_vulnerabilities}
\end{table*}

The results of verifying which kernel bugs were triggered under each
environment is illustrated in Table \ref{table:trigger_vulnerabilities}. 
We found that a substantial number of bugs were triggered in existing
virtualization systems. 
A full 35 out of 35 (100\%) bugs were triggered in Native Linux, 
while the other programs had somewhat lower rates: 14/35 (40\%) in
VirtualBox, 
11/35 (31.4\%)  in VMWare Workstation, 8/35 (22.9\%)  in Docker, and 8/35
(22.9\%) bugs in Graphene. 
In comparison, only 1 out of 35 bugs  (2.9\%)  was triggered in Lind. 
Comparing these results, Lind worked significantly better than the other
systems in limiting the triggering of kernel bugs.

We now discuss four vulnerabilities from Table \ref{table:trigger_vulnerabilities} 
in more detail. Those four bugs represent four different cases in which 
different system design philosophies have different security impacts. 

\begin{itemize}
\item \emph{Only Lind safe.}  Representative bug: CVE-2015-5706. As 
shown in our results, this vulnerability was triggered in every 
virtualization system we tested, and Native Linux, but not in Lind. This vulnerability 
is closely related to file system calls and file flags. It resides in the \texttt{fs/namei.c} 
kernel path. This bug can be triggered by making a \texttt{path\_openat()} function 
call with file flag \texttt{O\_TMPFILE}. \texttt{path\_openat()} will jump to the wrong 
place after \texttt{do\_tmpfile()}, and do \texttt{path\_cleanup()} twice. This will 
allow local users to perform use-after-free exploitation to cause a denial of service. 
This bug was not triggered in Lind, because Lind does not support the use of 
\texttt{O\_TMPFILE} file flag. In fact, the only call in the Repy sandbox that
opens a file does not take an argument for flags or other operations.  The
only arguments it takes are a filename (that must consist of a small number
of highly restricted characters) and a flag to indicate whether a file should
be created if it does not exist.
Other virtualization systems allow more complex configuration of flags to 
pass through to the underlying OS kernel. \cappos{Why does VirtualBox pass this
through?  My VirtualBox file system is a single VDI file.  How does this end
up calling into the host OS's kernel and why?}  \cappos{Does this vary 
if you use a different FS type on VirtualBox?}
In this case, the \texttt{O\_TMPFILE} file flag was 
allowed in Native Linux, VirtualBox, VMWare Workstation, Docker, and Graphene. Therefore, 
those systems suffer from the risk of this vulnerability.

\item \emph{All systems vulnerable.}  Representative bug: CVE-2014-4171. 
This is the only vulnerability in our test that was triggered in every 
system, including Lind. It resides inside the \texttt{mm/shmem.c} kernel path. This bug can 
be triggered by using \texttt{mmap()} system call to access a hole in the memory. 
The \texttt{mmap()} call then invokes \texttt{shmem\_fault()}, which will cause contention 
on \texttt{i\_mmap\_mutex}, and lead to a serious starvation.  
The reason that Lind also triggered this bug is because \texttt{mmap()} cannot easily 
be safely reimplemented inside our POSIX reimplementation.  The reason is
that \texttt{mmap()} sets up a memory region where the OS will later 
intervene and automatically convert all accesses into accesses to the
underlying file.  Thus the code does not explicitly make system calls.  As
a result, with Lind's design we cannot intercept those accesses in order
to cause them to call through the Repy sandbox kernel. As a result, 
Lind allows \texttt{mmap()} calls to directly access the kernel, which 
opens the chance to trigger this vulnerability. Similarly, in other 
virtualization systems we tested, \texttt{mmap()} is handled by the underlying
host OS kernel.  \cappos{is this really true?  What about in VirtualBox?
How does this happen when the underlying file isn't a unique file underneath?}
Therefore, this vulnerability was 
triggered in every system. 

\item \emph{Only Native Linux vulnerable.}  Representative bug: CVE-2014-5045. 
This vulnerability was only triggered inside Native Linux. It resides in the 
\texttt{fs/namei.c} kernel path. This vulnerability got triggered because 
the \texttt{mountpoint\_last()}
%\texttt{mountpoint\_last(struct nameidata *nd, struct path *path)} 
function does not properly 
maintain a reference count during attempts to use the \texttt{umount} system call, 
in conjunction with a symbolic link. Unmount on symbolic link could block another unmount operation, 
and allow attackers to cause a denial of service or deploy use-after-free exploitation. 
Lind's \yanyan{Lind's what?} does not implement, but similar functionality is implemented entirely 
within SafePOSIX.  Thus a bug would (at most) enable an attacker to execute
code within the Repy sandbox.
Other virtualization systems have their own metadata to maintain their file directories and symbolic links. \cappos{Why did the flag in the first example
work then?  Also, why doesn't this exist in Docker / LXC?}
Furthermore, symbolic links in those systems will be contained within the virtualization system's image, 
and will not be able to reach the underlying OS. In this case, those virtualization systems provide enough 
isolation to prevent this bug from happening. 

\item \emph{Some safe, some vulnerable.}  Representative bug: CVE-2014-8086. 
This vulnerability was not triggered inside Graphene and Lind, but was triggered inside 
VirtualBox, VMWare workstation, Docker, and Native Linux. It resides in the \texttt{sf/ext4/file.c} kernel path. \yanyan{\texttt{fs/ext4/file.c}?}
This bug can be triggered via making a file system write function call, together with \texttt{fcntl} function call 
with argument \texttt{F\_SETFL}, and \texttt{O\_DIRECT} flag. If triggered, it could allow attackers to cause 
a denial of service (file unavailability). Lind implements \texttt{fcntl}
in SafePOSIX, so the underlying kernel is not called.
Graphene checks and blocks certain system calls, including
a \texttt{fcntl} system call with the \texttt{O\_DIRECT} flag. 
Thus they both prevented this bug. Other systems like VirtualBox, VMWare Workstation, Docker, and Native Linux, 
all suffer from this vulnerability because they call into the host OS
kernel which handles the call.

\end{itemize}

As shown from the above four cases, a usual way to trigger a bug is to go through complex system calls, 
or basic system calls with complicated or rarely used flags. In Lind, our 
design philosophy is to reimplement complex system system call behavior
using simple and regularly used system calls with common settings.
Therefore, Lind only triggered 
1 out of 35 bugs.  The safely-reimplement design
has the least risk of triggering bugs in the underlying OS kernel.
%which is the main benefit of running applications in it.  

\subsubsection{Reachable Kernel Trace Analysis for Different Virtualization
Systems}
\label{Reachable-Kernel-Trace-Analysis-for-Different-Virtualization-Systems}

\begin{table}
\centering
\scriptsize
\caption{Reachable Kernel Trace Analysis for Different Virtualization
Systems}
\begin{tabular}{|l|l|l|l|}
  \hline
  \multirow{3}{1.5cm}{\bf Virtualization system} & \multicolumn{3}{c|}{\bf Kernel trace} \\ \cline{2-4}
  & \multirow{2}{1.5cm}{Compared to native Linux} & \multirow{2}{1.5cm}{In safe portion 
  (common paths)} & \multirow{2}{1cm}{In risky portion (uncommon paths)} \\
  & & & \\  \hline
  VirtualBox & 78.8 \% & 46.5 \% & 53.5 \% \\
  \hline
  \multirow{2}{1.5cm}{VMWare Workstation} & \multirow{2}{*}{72.6 \%} & 
  \multirow{2}{*}{50.2 \%} & \multirow{2}{*}{49.8 \%} \\ 
  & & & \\   \hline
  Docker & 61.3 \% & 58.4 \% & 41.6 \% \\
  \hline
  Graphene & 49.2 \% & 65.1 \% & 34.9 \% \\
  \hline
  Lind & 36.2 \% & 100 \% & 0 \% \\
  \hline
\end{tabular}
\label{table:trace-systems}
\end{table}

After establishing that Lind was efficient at preventing kernel bugs, we next 
examined how Lind was able to achieve this efficiency.  
We obtained the total reachable kernel trace for
each of the systems (including Lind) 
and further analyzed the components of those traces. These results
are shown in Table \ref{table:trace-systems}.

As seen in the table, Lind accessed the minimum amount of code in the OS
kernel. More importantly, 
all the kernel code it accessed was in the safe portion of the kernel, the
commonly used kernel paths. 
For example, a large portion of the kernel paths accessed by Lind lie in 
\texttt{fs/} to perform file system operations. 
In \texttt{fs/}, the commonly used paths that contain 
fewer bugs are the lines of code that do not involve complex function calls 
or complicated and rarely used arguments/flags. In Lind, only basic function calls, 
like \texttt{open()}, \texttt{close()}, \texttt{read()}, \texttt{write()}, \texttt{mkdir()}, 
\texttt{rmdir()}, are allowed. In addition, only commonly used flags are allowed. 
For example, for function \texttt{open()}, only 
\texttt{O\_CREAT}, \texttt{O\_EXCL}, \texttt{O\_APPEND}, \texttt{O\_TRUNC}, 
\texttt{O\_RDONLY}, \texttt{O\_WRONLY}, and \texttt{O\_RDWR} are permitted. 
%The use of only those basic system functions together with regularly used flags leads to 
%result that 
As a result, the reachable kernel trace we obtained with Lind all lies within the safe 
portion of the kernel, which contains fewer kernel bugs
as verified in \S{\ref{Verification-of-Hypothesis}}. 
%the safe portion of the kernel contains fewer kernel bugs. 
%So it make sense that Lind is less likely to trigger kernel bugs. 

The other virtualization systems all accessed a substantial number of code
paths in the kernel, 
and they all accessed a larger section of the risky portion. 
%the uncommonly used kernel paths. 
This is because they have 
more dependence on many complex system function calls, and 
allow extensive use of complicated flags. For example, 
Graphene provides a complex system call API that allows 
\texttt{fork()} and \texttt{signals}, which can access many risky lines of code. 
VirtualBox, VMWare Workstation, and Docker have even larger 
code base and more complicated system functions. They allow 
rarely used flags, such as \texttt{O\_TMPFILE}, \texttt{O\_NONBLOCK}, 
and \texttt{O\_DSYNC}, which can reach potentially dangerous lines 
of code.  
%
Based on our hypothesis, many historical bugs, as well as undetected
zero-day bugs, could be located there. 
Thus, accessing the risky portion without restriction is dangerous, and
leads to potential kernel bug exploitation. The results in Table 
\ref{table:trace-systems} verify our hypothesis.

To summarize, Lind triggers the least kernel bugs because 
%lies in the important fact that 
it has better control over the access to the OS kernel. 
Therefore, better results can be achieved with Lind, as a natural
outcome of its design.

\subsubsection{Reachable Kernel Trace Analysis for Repy Sandbox}
\label{Reachable-Kernel-Trace-Analysis-for-Repy-Sandbox}

An important question about Lind's security guarantee is what would happen if
there is a bug or a failure in Lind's TCB, 
the Repy sandbox kernel. Because the TCB has direct access to the OS
kernel, if a bug occurs in the TCB, 
it can potentially access the privileged OS kernel and trigger kernel bugs. 

To determine if a flaw in the TCB could endanger the kernel, 
we obtained the total reachable kernel trace in Repy and analyzed its
components. 
The results are shown in Table \ref{table:trace-Repy}. The trace of Repy is
slightly larger (5.8\%) than that of Lind.
This means that Repy's design can not allow attackers or bugs to 
have more access to the OS kernel, and only a small amount (5.8\%) of
additional OS kernel paths might be open. 
Those new kernel paths are added because some functions in Repy 
has more capabilities than the system call interfaces 
provided by Lind. For example, in Repy, 
%\texttt{sendmessage(destip, destport, message, localip, localport)} and 
%\texttt{openconnection(destip, destport, localip, localport, timeout)}
\texttt{sendmessage()} and \texttt{openconnection()}
functions could reach out to more lines of code when fuzzed.  

However, the kernel trace of Repy still lies completely within the safe
portion of the OS kernel. 
Since the safe portion contains fewer kernel bugs, the Repy sandbox kernel
will have a very slim chance to trigger OS kernel bugs.

\cappos{I think this confuses bugs in SafePOSIX with that of the sandbox
kernel.  I wrote the latter text in a subsection below...}
The results explained above shows that even if our Repy sandbox kernel has a
bug or failure inside, 
it only slightly increases the amount of OS kernel paths open to attacks,
and all these paths accessed are still inside the safe portion. 
Therefore, Repy will not grant attackers more opportunities to trigger OS
kernel bugs. 
Since Repy, arguably the main security weakness of the system, can be
considered safe through our analysis, 
it shows that Lind can provide strong security to run user applications.

\begin{table}
\centering
\scriptsize
\caption{Reachable Kernel Trace Analysis for the Repy Sandbox. }
\begin{tabular}{|l|l|l|l|l|}
  \hline
  \multirow{3}{.8cm}{\bf Sandbox} & \multicolumn{4}{c|}{\bf Kernel trace} \\ \cline{2-5}
  & \multirow{2}{1cm}{Compared to Lind} & 
  \multirow{2}{1.3cm}{Compared to native Linux} & \multirow{2}{1.7cm}{In safe portion 
  (common paths)} & \multirow{2}{1.9cm}{In risky portion (uncommon paths)} \\
  & & & & \\  \hline
  
  Repy & 105.8 \% & 38.3 \% & 100 \%  & 0 \%  \\
  \hline
\end{tabular}
\label{table:trace-Repy}
\end{table}


\subsubsection{Repy Sandbox Kernel}
\label{Sandbox-Kernel-Bugs}
\cappos{Possibly move this to the implementation...}

A natural concern with any sandbox design is that bugs are simply pushed into
another part of the trusted code base.  As it is the only piece of code added
to the system call paths of the TCB, the Repy sandbox kernel's security is of
paramount concern.

The sandbox kernel consists of only XXXX lines of code \cappos{continue...}.
The code is written to provide straightforward access to the minimal set
of the system call API needed to build general computational funcationality.
The code was written using code style guidelines that intend to ease the
process of security audit of the code~\cite{style}.

The sandbox kernel code has been 
audited by a professional penetration tester.  Since 2010, there has also been
a bug bounty program for security flaws in the sandbox. 
The code is deployed in daily use across thousands of devices, 
including on the Seattle testbed \cite{seattle}, and has been examined by 
hundreds of parties.  Developers have reported
XXX issues for problems in other parts of the systems. However, to date
no security flaws have been found in the sandbox kernel.
This does not provide any strong guarantees that bugs could not exist.  If bugs do
exist in this system, the security of the system could be compromised.  
However, having a small, easily auditable piece of code helps to reduce the
risk of this occurrence.

\subsubsection{Performance Evaluation}
\label{Performance-Evaluation}

\begin{figure}
\centering
\includegraphics[width=1.0\columnwidth]{diagram/lind_oakland16_performance.pdf}
\caption{Applications Runtime Performance: Native Linux vs. Lind}
\label{fig:performance_applications}
\end{figure}

While efficiency was not our main goal, we evaluated Lind to see 
its performance compared to other systems.  Note that we did not performance
optimize Lind in any way.

We first compiled and ran three widely used applications, 
a primes calculator, GNU \texttt{grep}, and GNU \texttt{wget}. All ran unaltered and
correctly inside Lind. The source code of each of the applications remained
unmodified. To run them, it was sufficient to just recompile the source code.
This uses NaCl's compiler and uses Lind's \texttt{glibc} to call
into SafePOSIX.
Figure \ref{fig:performance_applications} shows the runtime performance
results. 
The primes application ran in Lind has a 6\% performance overhead compared to 
Native Linux. CPU bound applications, like the primes, generally experience small overhead, 
because they run only inside the NaCl computation sandbox. No system calls are required, 
and there is no need to go through the safe POSIX interface. The small amount of overhead 
is generated by NaCl's instruction alignment at building time. Another reason for the overhead 
is that the instructions built by NaCl have a higher rate of cache misses, which can slowdown the 
program. 
We expect other CPU bound processes to behave similarly. 
\texttt{grep} experienced roughly 11x slowdown over Native Linux, while \texttt{wget}
slowdown was roughly 19x. Since they are both I/O heavy applications, 
each call repeatedly calls into the SafePOSIX code which then reimplements
the call.  The additional computation of SafePOSIX produced the additional
overhead.  

Next, we ran the Tor router software in Lind \cappos{what version}. Tor simply
needs to be recompiled to run in Lind. 
We used the benchmarks that come with Tor to test its common operations. 
A summary of the results is shown in Table \ref{table:performance_tor}. The
benchmarks focus on cryptographic operations, 
which are CPU intensive, but also make system calls like \texttt{getpid}, and reads to
\texttt{/dev/urandom}.
The digest operations time the access of a map of message digests. 
The AES operations time AES encryptions of several sizes and message 
digest creation. 
Cell processing executes full packet encryption and decryption. In our
test, Lind slowed down these operations by 2.5x to 5x. We believe these
slowdowns are due to the increased code size produced by NaCl, 
\cappos{I'm not sure why this would be.  Does NaCl show this too?}
and the
increased overhead from Lind's safe POSIX system call interface. 

As shown above,  Lind incurs some performance overhead in most cases. 
It should be noted that, we have not  yet attempted to optimize its performance. 
However, since an attack on the kernel has devastating
consequences, %at this initial stage, 
a tradeoff between security and performance is well justified. 
The fact that Lind is able to run many \cappos{4 applications isn't many...
Do we have Apache numbers or something else to quantify?} legacy applications 
suggests that it
is a positive step towards building new secure systems. 

\begin{table}
\centering
\scriptsize
\caption{Performance Results on Tor's Built-in Benchmark Program: Native
Linux vs. Lind.}
\begin{tabular}{|r|r|r|r|}
  \hline
  {\bf Benchmark} & {\bf Native Code} & {\bf Lind} & {\bf Impact}  \\
  \hline
  Digest Tests: & & & \\
  Set & 54.80 nsec/element & 176.86 nsec/element & 3.22x \\
  Get & 42.30 nsec/element & 134.38 nsec/element & 3.17x \\
  Add & 11.69 nsec/element & 53.91 nsec/element & 4.61x \\
  IsIn & 8.24 nsec/element & 39.82 nsec/element & 4.83x \\
  \hline
  AES Tests: & & & \\
  1 Byte & 14.83 nsec/B & 36.93 nsec/B & 2.49x \\
  16 Byte & 7.45 nsec/B & 16.95 nsec/B & 2.28x \\
  1024 Byte & 6.91 nsec/B & 15.42 nsec/B & 2.23x \\
  4096 Byte & 6.96 nsec/B & 15.35 nsec/B & 2.21x \\
  8192 Byte & 6.94 nsec/B & 15.47 nsec/B & 2.23x \\
  Cell Sized & 6.81 nsec/B & 14.71 nsec/B & 2.16x \\
  \hline
  Cell Processing: & & & \\
  Inbound & 3378.18 nsec/cell & 8418.03 nsec/cell & 2.49x \\
  (per Byte) & 6.64 nsec/B & 16.54 nsec/B & - \\
  Outbound & 3384.01 nsec/cell & 8127.42 nsec/cell & 2.40x \\
  (per Byte) & 6.65 nsec/B & 15.97 nsec/B & - \\
  \hline
\end{tabular}
\label{table:performance_tor}
\end{table}

