\section{Implementation of the Lind Prototype}
\label{sec.implementation}

Based on the design introduced in Section \S{\ref{sec.design}}, 
we implemented a secure sandbox system, Lind, 
for running untrusted user programs on vulnerable OS kernels. 
Lind adapts two basic technologies as its building blocks\textendash 
Google's Native Client, and Seattle's Repy. 

As described in \S{\ref{sec.design}} (Figure \ref{fig:design}), our design has
two main components\textendash a computation module that isolates the 
application and a system API provided by a sandbox that isolates the complex 
portions of POSIX.
The Google's Native Client (NaCl) ~\cite{NaCl-09} served as the computation 
module 
and Seattle's Repy as the system API module (Figure \ref{fig:architecture}).
\cappos{What is this saying?  It is repetitive...}

\begin{figure}%[h]
\centering
\includegraphics[width=1.0\columnwidth]{diagram/lind_architecture_new.png}
\caption{Architecture of Lind including various components such as NaCl, Nacl glibc, and Repy Sandbox. User level application issues system calls that are dispatched through the Repy OS connector that bridges the Lind system to the OS Kernel.}

\label{fig:architecture}
\end{figure}

\subsection{The Computation Module of Lind: Google's Native Client}

We utilize NaCl in our design to isolate the computation of the user application.  NaCl is a widely used environment for efficient  execution of legacy code in the form of x86 and ARM binaries, which allows Lind to work on most types of legacy code. Programs are compiled with the NaCl compiler which produces a binary with software fault isolation. This prevents the majority of the application from performing system calls  or executing arbitrary instructions. 

To perform a system call, the application will call into a small privileged
part of the process that is part of the NaCl TCB which forwards system calls.
 In NaCl, these calls are usually either passed to the OS or to Chrome for 
processing \cappos{Check}.  To build Lind, we changed the NaCl TCP to 
instead forward these calls to the SafePOSIX implementation
for processing.  The NaCl glibc module contained stubs that rejected operations
that Chrome would not handle.  We added functionality to NaCl's glibc so that
those calls were forwarded into SafePOSIX for processing.
\cappos{Need to compare NaCl execution w/ Chrome possibly
with fuzzing to others in the eval}  


%NaCl can perform the functions required by the computation module well, and it is easy to 
%connect with our system API module because NaCl uses glibc to perform system calls. 
%Modification to NaCl's glibc would allow us to redirect those system call requests to our own system API module. 
%
%NaCl is a sandbox used to execute untrusted x86 native code. 
%It aims to give applications the computational performance of native applications without compromising safety. 
%NaCl uses software fault isolation and a secure runtime to direct system interaction and 
%side effects through interfaces managed by the program. It provides operating system portability 
%for binary code while supporting performance-oriented features, such as thread support, 
%instruction set extensions, such as SSE, and use of compiler intrinsics and a hand-coded assembler. 
%It also allows the efficient execution of legacy code in the form of x86 and ARM binaries 
%that are built with a lightly modified compiler tool chain. 

\subsection{The System API Module of Lind: Seattle's Repy}

To build an API to access the safe parts of the underlying kernel, we need 
two things.  First, we need a restricted sandbox that isolates computation
and only allows access to commonly used kernel paths.  We used 
Seattle's Repy~\cite{Repy-10} sandbox to perform this task.
Second, we need to build a POSIX implementation that runs within that sandbox.  

%pplications. For Lind, we used Repy to build our system API module. 
%To be more specific, our system API module has a very small sandbox kernel 
%as TCB,  written with Python. On top of the sandbox kernel, 
%we use Repy code to safely reimplement complex system functions.

\subsubsection{The Repy Sandbox Kernel}

As discussed earlier in \S{\ref{sec.design}}, the sandbox kernel needs to be secure and bug-free. 
Because it is the TCB of the system, any bugs within it could cause fatal problems, 
and allow attackers to access the OS kernel and gain kernel privilege. 
We used Seattle's Repy system API module due to its tiny sandbox kernel ( 
comprised of only around 8,000 LOC). Using Repy also was utilized according to our key design principle that 
our proposed system should only access the safe portions of the OS kernel. In the Repy sandbox kernel, 
there are 33 basic API functions, including 13 network functions, 6 file functions, 6 threading functions, 
and 8 miscellaneous functions (Table \ref{table:RepyKernel}) \cite{Repy-10}, \cite{RepyKernel}. Most of those functions are simple and 
regularly used system calls that access the commonly used kernel paths. 


\begin{table}
\centering
\caption {Repy sandbox kernel capabilities that it
provides to support the Nacl functions such as networking, file I/O operations and threading.}

  \begin{tabular}{ | p{2.5cm} | p{4.5cm} |}
 \hline
  \textbf{NaCl Function} & \textbf{Available System Calls}  \\ \hline
    
Networking & \emph{gethostbyname, openconnection, getmyip, socket.send, socket.receive, socket.close, listenforconnection, tcpserversocket.getconnection, tcpserversocket.close, sendmessage, listenformessage, udpserversocket.getmessage, and udpserversocket.close.} \\ \hline
 
I/O Operations & \emph{openfile, file.close, file.readat, file.writeat, listfiles, and removefile.} \\ \hline

Threading & \emph{createlock, sleep, lock.acquire, lock.release, createthread, and getthreadname.} \\ \hline   

Miscellaneous Functions & \emph{getruntime, randombytes, log, exitall, createvirtualnamespace, virtualnamespace.evaluate, getresources, and getlasterror.}  \\ \hline
    \end{tabular}
    \label{table:RepyKernel}
\end{table}

%\begin{table}
%\centering
%\scriptsize
%\caption {System Functions in the Repy Sandbox Kernel.  \cappos{Need to
%clearly explain the takeaway.}}
%\begin{tabular}{|l|}
%  \hline
% \textbf{Network Functions} \\
%  \hline
%  gethostbyname(name) \\
%  \hline
%  getmyip() \\
%  \hline
%  openconnection(destip, destport, localip, localport, timeout) \\
%  \hline
%  socket.close() \\
%  \hline
%  socket.recv(numbytes) \\
%  \hline
%  socket.send(message) \\
%  \hline
%  listenforconnection(localip, localport) \\
%  \hline
%  tcpserversocket.getconnection() \\
%  \hline
%  tcpserversocket.close()\\
%  \hline
%  sendmessage(destip, destport, message, localip, localport) \\
%  \hline
%  listenformessage(localip, localport) \\
%  \hline
%  udpserversocket.getmessage() \\
%  \hline
%  udpserversocket.close() \\
%  \hline \hline
%  \textbf{File Functions} \\
%  \hline
%  openfile(filename, create) \\
%  \hline
%  file.close() \\
%  \hline
%  file.readat(sizelimit, offset) \\
%  \hline
% file.writeat(data, offset) \\
%  \hline
%  listfiles() \\
%  \hline
%  removefile(filename) \\
%  \hline \hline
%  \textbf{Threading Functions} \\
%  \hline
%  createlock() \\
%  \hline
%  lock.acquire(blocking) \\
%  \hline
%  lock.release() \\
%  \hline
%  createthread(function) \\
 % \hline
 % sleep(seconds) \\
  %\hline
  %getthreadname() \\
  %\hline \hline
  %\textbf{Miscellaneous Functions} \\
  %\hline
 % getruntime() \\
  %\hline
 % randombytes() \\
  %\hline
  %log(*args) \\
  %\hline
  %exitall() \\
  %\hline
  %createvirtualnamespace(code, name) \\
  %\hline
  %virtualnamespace.evaluate(context) \\
  %\hline
  %getresources() \\
  %\hline
  %getlasterror() \\
  %\hline
%\end{tabular}
%\label{table:RepyKernel}
%\end{table}

\subsubsection{The SafePOSIX Reimplementation}

The key responsibility of our system API module is to serve system call requests from user code. 
In the Lind system, those system call requests are issued from the user code, 
received by the computation module NaCl, and then redirected to our system API module. 
The API module includes a POSIX API to serve those requests. 

A POSIX API is a set of standard operating system interfaces that provide operating system functions 
to the user code. However, the POSIX API is large and complex enough to ensure that its implementation is secure and bug-free. 

Our choice to use Repy helped us solve this architectural security problem. 
Since Repy is a programming language sandbox, it can provide the ideal isolation 
we needed when constructing our POSIX API. In Lind, 
complex system functions are reimplemented using Repy code, 
based on the ``safe-reimplement'' principle from our design in \S{\ref{sec.design}}.

\subsection{Operations}

This section discusses the logical connection between different pieces of the Lind approach including NaCl and Seattle's Repy sandbox.

Lind combines the NaCl and Repy component to provide native computation and 
safe access to the system. Untrusted programs are run in NaCl, 
but access to all system resources is diverted to a Repy program. 
This program is responsible for accessing the system on behalf of the program, 
which is called the Lind library OS. A NaCl sandbox is built on top of the Repy sandbox. 

To service a system call in NaCl, a server routine marshals its arguments into a text string, 
and sends the call and the arguments to Repy sandbox. 
The library OS then executes the appropriate system call, marshals the result and 
returns it back to NaCl. The result is eventually returned as the appropriate native type to the calling program. 

Lind is designed to minimize its modifications within the TCB of these two sandboxes. 
By running the Lind code is run from within the two sandboxes, 
the modifications to the sandboxes themselves (and therefore the TCB) were extremely small. 

The only complex part of Lind is the library OS, which runs in Repy. 
However, because Python is a very powerful language that provide rich functions, 
it significantly simplified the construction of Lind. Even though Python is considered ``slow'' by some, 
the internals of an application in Lind are run in NaCl, a very high performance environment. 
This balances the performance of the system, with the ease of implementation and maintenance 
of the library OS component of Lind. 

The novel idea behind the architecture and design for sandboxing using Lind ensures the programs are portable. 
Programs running inside Lind are written to work against a standard POSIX glibc interface, 
and, since the Lind runtime is strictly user-level, it can work on many different platforms 
including Linux, Mac OS X and Windows.

The lightweight and minimized overhead is another feature of Lind. Because the sandbox only incurs overhead when there is a system call, 
overall overhead for Lind is low. Lind uses a native interface for execution, 
allowing CPU-and-memory-intensive applications to run at speeds that are equivalent to NaCl and near native speed. 

\cappos{Does this fit better in a discussion section?  Why is it in the
implementation?}
\subsection{The Benefits of the Two Sandbox System}

In reviewing the design of Lind, some fundamental questions could be raised about the dual sandbox design. 
Why is one not enough? And, if two is good, why not use three or more? 
The answer to the first question is that the kernel interface is extremely rich and hard to protect. 
In order to have minimal impact on the kernel, as well as provide sufficient API for legacy applications, 
we need to have one sandbox focuses on protecting the kernel and providing POSIX API, 
while a second sandbox focuses on efficiently executing applications. 
So our approach includes two sandboxes, one as the computation module, 
and the other one as system API module. As to the second question, 
which could be rephrased as ``why not sandbox Repy TCB and get more security?,'' 
the lowest level sandbox eventually must have some fundamental, 
even if  limited access to system resources, such as memory, and storage, threads. 
So even if we were to sandbox Repy TCB and have additional sandboxes, 
the one at the bottom level will still access the OS kernel in a similar way. 
Thus, having multiple sandboxes does not provide any extra security benefits. 

In \S{\ref{sec.evaluation}}, we present how the dual sandboxes and other elements of Lind performed in 
head-to-head evaluation against other virtualization systems.
