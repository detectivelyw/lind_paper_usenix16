\section{Related Work}
\label{sec.related_work}
This section summarizes other kernel security approaches, such as virtualization
 or other isolation mechanisms
that aim to ensure the safety of privileged code in user space and kernel space.

%\lois{Do all of your subtopics fall into this one category? If not, this introductory sentence should be modified}

\subsection{Virtualization}

\textbf{Language-based virtualization.}
Programming languages that provide safety through virtualization such
as Java, JavaScript, Lua~\cite{Lua}, and
Silverlight~\cite{Silverlight} are commonly used in application-level
sandboxing. These safe languages provide virtualized environments to
check the safety of the running code by a monitor process. They
combine untrusted application code with an interpreter and
standard libraries that consolidate routines to perform I/O, network
communication, and other sensitive functions.
%For example, the Java
%Virtual Machine (JVM) \cite{JVM} functions as an application-level
%sandbox to separate untrusted code from the OS in addition to
%performing safety checks for avoiding unauthorized branching in
%memory.
%
There are also sandboxing solutions based on type-safety of programming
languages, i.e. validating through a type-checker \cite{JS-Sandboxing}
or enforcing security policies on an untrusted system through a
reference monitor \cite{JS-Sandboxing1}. %Compared to these approaches,
%the Lind protection model ensures stronger security than just executing the code in a browser's sandbox to isolate untrusted programs.

%Lind enforces strict policies and
%rules to validate all segments of a code before executing it through a
%dual-layer protection.

Though many sandboxes implement the bulk of standard libraries in
memory-safe languages like Java or C\#, flaws in this code can
still pose a threat. In fact, many security critical bugs can be found
in the standard libraries \cite{JavaBugs, Java-Lessons}.
%Researchers have also
%discovered many severe bugs in Java \cite{Java-Lessons}.
Any bug or failure in a programming language virtual
machine is usually fatal. In contrast, Lind with a very small TCB
 (approximately
8,000 LOC) enhances security of privileged code compared to the above. \lois{why?}
%\lois{OK, what does the TCB have to do with these languages?}

%because of the POSIX implementation with Repy to build the sandbox.

\textbf{OS virtualization.}
OS virtualization techniques can be divided into two categories:
%Type-I or Type-II virtualization. Examples of Type-I virtualization
\textit{bare-metal hardware} virtualization, such as VMware ESX Server, Xen,
LXC~\cite{LXC}, BSD’s jail, Solaris zones, and Hyper-V, and
\textit{hosted hypervisor} virtualization such as VMware
Workstation, VMware Server, VirtualPC and the open-source counterpart
VirtualBox.
%
Security by isolation \cite{Qubes}, \cite{Overshadow},
\cite{SecureVM}, \cite{HypSec} is a feature of OS virtualization that
provides safe executing environments through containment for multiple
user-level virtual environments that share the same hardware. This
approach relies on the VMM to confine untrusted applications within
guest OSs. However, there are limitations due to
the large attack vectors against the hypervisors, including
vulnerabilities of software and configuration risk.
%For example, the
%small code footprint of a hypervisor can become a potential
%vulnerability and lead to serious problems. Maintaining two OSs in
%virtulized environments is among the other issues that add complexity
%\cite{Virt-Issues}. Attackers are able to exploit vulnerabilities to
%escape these systems and access arbitrary code in the underlying host
%systems.
Lind deals with these concerns by using a much smaller and safer TCB.

\textbf{Library OS.}
Library OSes allow applications to efficiently
obtain the benefits of virtual machines,
including security isolation, host platform compatibility, and
migration. Libra \cite {Libra} is an example of a library OS that provides applications the possibility to customize their OSs within VMs.
This solution is similar to the exokernel approach where a small trusted kernel (exokernel) is separated from the untrusted libraries
that function as an abstraction layer between the applications and the underlying hardware.
Libra utilizes a hypervisor instead of the exokernel to deliver efficient portability and performance.

Drawbridge \cite{Drawbridge-11} is another approach that uses picoprocesses (lightweight containers) security monitor (to enforce rules)
and a library OS to present a Windows persona for a wide variety of Windows applications.Similar to Lind,
it restricts access from usermode to host OS through a number of operations that pass through the security monitor.
%\lois{How does this fit with Lind?}

%This approach brings many of the benefits of VM based temporal,
%spatial and fault isolation properties to a per-process level.

Bascule \cite{Bascule}, an architecture for library OS extensions
based on Drawbridge, allows application behavior to be customized by
extensions loaded at runtime.
Graphene \cite{Graphene-14} is a recent library OS system that
seamlessly and efficiently executes both single and
multi-process applications, with low memory and performance overheads.
%It broadens the library OS paradigm to support secure, multi-process
%APIs, such as copy-on-write fork, signals, and System V IPC.
Haven \cite{Haven} uses a library OS to implement
shielded execution of unmodified server applications
in an untrusted cloud host.
%Haven leverages the hardware protection of
%Intel SGX to defend against
%privileged code and physical attacks, such as memory probes, but also
%addresses the dual challenges of
%executing unmodified legacy binaries and protecting them from a
%malicious host.
Its library OS technique is similar to Lind but differs in
that existing library OS systems rely heavily on
the underlying kernels to perform system functions. In contrast, Lind only
relies on a very limited set of system functions,
and reconstructs most OS functions with its own safe Repy code.

\subsection{System Call Interposition}

System call interposition (SCI) provides
%offers a number of properties that make it attractive to run applications
secure execution of applications by exposing a minimal kernel surface.
This approach usually ensures the userspace policies through a monitor process that
interacts with the system call execution engine.
%for building sandboxes, though it can be error prone
\cite{SCI-04}.
%Approaches for delegation and filtering have been extensively studied. For example,
%along with their respective tradeoffs between security and
%performance.
Janus (J2) \cite{Janus0:96, Janus:99} is an SCI solution for confining untrusted applications that includes system call filtering
and sandboxing mechanisms.Garfinkel et al. in another effort, introduces Ostia \cite{SCI-04}, where system calls are delegated to
agents that enforce the policies. Ostia resolves the tight dependencies with the underlying OS through process emulation and
the agents model at the userspace. Other issues in utilizing SCI solutions include the difficulty of appropriately replicating OS semantics,
multi-threaded applications that cause race conditions, or indirect paths to resources that are often overlooked.
In addition, system calls that are denied can cause inconsistencies that are discussed in \cite{Problems-SCI}.

Recently, Wang et al. \cite{Jitk} discuss the issues of traditional SCI systems
in the context of in-kernel interpreters such as Seccomp, which is used by several
security critical applications, including Chrome, OpenSSH and Tor.
The authors propose Jitk architecture to avoid the weaknesses in traditional
kernelspace system call interceptors that arise from the unforeseen bugs within
the interceptors. Jitk provides a formal verification tool that ensure the
correct translation of the userspace submitted code, e.g. the  the BSD packet filter and INET-DIAG.

%The kernel module enforces policy, denies direct access to restricted
%resources, and delegates certain calls to an emulation library.
%which sends transformed system calls to the agents.
%The agent reads the policy file and handles the delegation of calls.


%Nevertheless, this technique is very useful and has inspired many new techniques, such as library OSes.
%The concepts behind system call interposition have evolved into other modern techniques and has benefited many security systems, including Lind.

The SCI approach is very similar to the Lind isolation mechanism but a key difference is the actual execution
of a system call through the Lind's reimplementation to ensure safe POSIX API for the userspace applications.

\subsection{Software Fault Isolation}
Software fault isolation (SFI) is an alternative to hardware memory
protection when running two applications in one address space. SFI provides sandboxing in which native
instructions can only be executed if they do not violate the sandbox's
constraints \cite{SFI:93}, and security policies are enforced through machine-level
code analysis. In this approach, memory
writes are protected and code jumps cannot access predefined memory of
other programs to execute their codes.

The preliminary design of SFI was build on RISC
architectures. The authors of PittSFIeld \cite{PittSFIeld} optimized and
extended the original SFI to support CISC architectures.
%For this purpose, the source instructions are padded with no-ops to fit, i.e.
%in the 16-byte x86 byte chunk alignment, where a call instruction is
%appended. A sequence of instructions form a instruction streams that
%ensures execution order of the sequence. The final code before
%execution will be checked by a verifier component to ensure safety.
%
SFI has been also used in MisFIT \cite{MISFit} to ensure kernel
module integrity for x86. The authors emphasized the
extendability of object-oriented programming languages to eliminate
the need for remote calls to achieve high throughput.
%
Nooks \cite{Nooks:03} is another SFI-based solution that provides
a protected environment for running device drivers by isolating kernel
modules. Nooks' runtime environment is located within the
kernel and it includes the majority of drivers that needs to be protected
from each other.
These SFI containers embody the extensions to demonstrate the integrity
validation check in conjunction with the host environment resources.
%i.e. network nook and video nook are two protection
%domains without the possibility for memory writes outside their
%protection domain. The Nook layer function as a reference monitor
%between device devices and physical hardware by forwarding interrupts
%Nooks also wrap calls from the operating system kernel into device
%drivers and from device drivers into the kernel, allowing the
%operating system to track resource usage and verify data that is
%passed into and out of the kernel. Object tracking to allow OS for
%resource consumption is another feature of wrappers in Nooks.

%To enforce the safety constraints, programs demand a specific compiler
%in advance to validate and load the programs. For example, in
Another SFI system called Byte
Granularity Isolation (BGI) \cite{Castro-BGI} was introduced as a runtime tool to isolate drivers in
separate protection SFI domains, while sharing the same address space
between different domains.
%BGI mainly ensures write-integrity,
%control-flow integrity and type safety for kernel objects. For this
%purpose, a BGI compiler is required to transform the unmodified driver
%source code to instrumented driver. The instrumented driver will be
%linked to the BGI interposition libraries that enforce the protection
%constraints to produce the BGI drivers. The protection constraints
%that are enforced by the interposition library are implemented in
%access control lists (ACL). The ACLs contain policies for each byte of
%virtual memory and domain permissions.
A major issue of BGI is that it
does not solve the vulnerabilities at kernel-levels.

Recently, Google provided Native Client (NaCl) \cite{NaCl-09} for
Chrome browser to allow native executable code to be run directly in a
browser using the PittSFIeld semantics. NaCl prevents suspicious code
from memory corruption or direct access to the underlying system
resources. For this purpose NaCl loads untrusted modules from the
trusted modules into two different address spaces, where in most SFI
approaches both untrusted and trusted codes are loaded into a common
address space.

%Mao et al. \cite{LXFI} study the shortcomings of XFI \cite{XFI} and
%BGI \cite{Castro-BGI} kernel module isolation mechanisms that separate
%kernel modules from the core kernel. Integrity of API calls was one of
%the challenging issues in existing solutions. For example in XFI there
%were no argument validation. LXFI solution was proposed as an
%extension of existing SFI solutions with language annotation. LXFI
%added argument integrity in XFI and call back integrity in BGI in
%addition to enabling programmers for specifying principals for access
%permissions within a module.
%
%In another effort \cite{PSFI}, Kroll et al. discuss the issues of
%machine-level SFI that causes portability limitations across different
%hardware architectures. The authors propose portable SFI through a
%higher level of abstraction at the compiler layer. To this end,
%programs are rewritten by an intermediate-level language compiler to
%comply with policies.
