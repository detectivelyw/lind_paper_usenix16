\section{Conclusion}
\label{sec.conclusion}

%The idea of isolating untrusted user applications from the underlying privileged code
%to avoid its exploitation by bugs has been realized in different implementations,
%but there is no standard method for creating this isolation.
%In addition, experience suggests that isolation by itself does not guarantee the security of a system.
%
In this paper, we proposed a new metric based on quantitative measures derived from
the execution of kernel code when running user applications.
We verified the hypothesis that commonly used kernel paths contain fewer bugs.
%Using our metric, we generated findings that suggest the hypothesis is reasonable,
% and it become the key principle behind a new design for building secure systems,
Our metric was used to implement a new virtualized security system called Lind. Designed with a minimized
TCB and interacting with the kernel in only commonly used paths, Lind addresses the need to 
support risky system calls by securely reconstructing complex, yet essential OS functionality inside a sandbox.
%
Evaluation results have shown that Lind is the least likely to trigger zero-day Linux kernel bugs,
when compared to seven other virtualization systems, such as VirtualBox, VMWare Workstation, Docker, LXC, 
QEMU, KVM and Graphene.
%This suggests that systems using our design are likely to be more secure.

All of the data and source code for this paper is available at our Lind website~\cite{Lind}. 
For further information and questions, please contact the authors. 
%For access to the kernel exploit code created in this study, please contact the
%authors.